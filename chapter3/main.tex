\documentclass[a4paper,11pt,fleqn,dvipdfmx]{article}

\usepackage[top=20mm,bottom=20mm,left=20mm,right=20mm]{geometry}

\usepackage{amsmath, amssymb, tikz}
\newcommand{\ansja}[1]{\noindent\textbf{解答 #1}\\[2mm]}
\newcommand{\ansen}[1]{\noindent\textbf{Ans #1}\\[2mm]}
\newcommand{\maru}[1]{\hspace{-0.25em}\raisebox{-0.7ex}[0ex][0ex]{
    \tikz{
        \draw (0,0) circle (0.18cm);
        \draw (0,0) node {#1};
    }
}\hspace{-0.25em}}
\newcommand{\qed}[0]{\hfill\fbox{}}
\newcommand\Tau{\mathcal{T}}
\newcommand{\kakko}[1]{\raisebox{.2ex}{(}#1\raisebox{.2ex}{)}}
\newcommand{\syomei}{\raisebox{-0.8ex}[0ex][0ex]{\tikz{\draw[line width=0.8pt, rounded corners=2pt] (0,0) -- (0.69,0) -- (0.69,0.36) -- (0,0.36) -- cycle;\draw[line width=0.3pt] (0.345,0.17) node {\textcolor{black}{\footnotesize \bf 証明}};}}\quad}
\newcommand{\syomeien}{\raisebox{-0.8ex}[0ex][0ex]{\tikz{\draw[line width=0.8pt, rounded corners=2pt] (0,0) -- (1.1,0) -- (1.1,0.36) -- (0,0.36) -- cycle;\draw[line width=0.3pt] (0.535,0.17) node {\textcolor{black}{\footnotesize \bf Proof}};}}\quad}

\begin{document}

    \ansja{3.2.4}
        \begin{align*}
            n(S_0) &= 0 \\
            n(S_1) &= 3 + 3 \times 0 + 0 ^ 3 = 3 \\
            n(S_2) &= 3 + 3 \times 3 + 3 ^ 3 = 39 \\
            n(S_3) &= 3 + 3 \times 39 + 39 ^ 3 = 59439
        \end{align*}

    \vspace{10mm}

    \ansja{3.2.5}
        任意の $i$ について $S_i \subseteq S_{i+1}$ であることを数学的帰納法で示す.
        \begin{itemize}
            \item $i=0$ のとき \\[2mm]
                $S_0 = \emptyset$ より, $S_0 \subseteq S_1$ である.
            \item $i=1$ のとき \\[2mm]
                $S_1 = \{\texttt{true}, \; \texttt{false}, \; 0\}$ より,$S_1 \subseteq S_2$ である.
            \item $i=k$ のとき,$S_i \subseteq S_{i+1}$ が成り立つと仮定して,$i=k+1$ のとき \\[2mm]
                $t_1 \in S_{k+1}$ について
                \begin{itemize}
                    \item $t_1$ が定数のとき \\[2mm]
                        $S$ の定義より,$t_1 \in S_{k+2}$ である.
                    \item $t_1 = \texttt{succ } t_2$ または $t_1 = \texttt{pred } t_2$ または $t_1 = \texttt{iszero } t_2$ のとき \\[2mm]
                        $S$ の定義より,$t_2 \in S_k$ であり,帰納法の仮定から $t_2 \in S_{k+1}$ を得る. \\
                        ゆえに,$S$ の定義から $t_1 \in S_{k+2}$ である.
                    \item $t_1 = \texttt{if } t_2 \texttt{ then } t_3 \texttt{ else } t_4$ のとき \\[2mm]
                        $S$ の定義より,$t_2, t_3, t_4 \in S_k$ であり,帰納法の仮定から $t_2, t_3, t_4 \in S_{k+1}$ を得る. \\
                        ゆえに,$S$ の定義から $t_1 \in S_{k+2}$ である.
                \end{itemize}
                したがって,$S_{i+1} \subseteq S_{i+2}$ が成り立つ.
        \end{itemize}
        以上から,各 $i$ について $S_i \subseteq S_{i+1}$ であるので,集合 $S_i$ は累積的である.

    \pagebreak

    \ansen{3.2.5}
        We prove that $S_i \subseteq S_{i+1}$ holds for all $i$ by mathematical induction.
        \begin{itemize}
            \item For $i=0$ \\[2mm]
                Since $S_0 = \emptyset$, $S_0 \subseteq S_1$ holds.
            \item For $i=1$ \\[2mm]
                Since $S_1 = \{\texttt{true}, \; \texttt{false}, \; 0\}$, $S_1 \subseteq S_2$ holds.
            \item Assume $S_i \subseteq S_{i+1}$ holds for $i=k$, then for $i=k+1$ \\[2mm]
                For $t_1 \in S_{k+1}$
                \begin{itemize}
                    \item When $t_1$ is a constant \\[2mm]
                        By definition of $S$, $t_1 \in S_{k+2}$.
                    \item When $t_1 = \texttt{succ } t_2$ or $t_1 = \texttt{pred } t_2$ or $t_1 = \texttt{iszero } t_2$ \\[2mm]
                        By definition of $S$, $t_2 \in S_k$, and by induction hypothesis, we get $t_2 \in S_{k+1}$. \\
                        Therefore, by definition of $S$, $t_1 \in S_{k+2}$.
                    \item When $t_1 = \texttt{if } t_2 \texttt{ then } t_3 \texttt{ else } t_4$ \\[2mm]
                        By definition of $S$, $t_2, t_3, t_4 \in S_k$, and by induction hypothesis, we get $t_2, t_3, t_4 \in S_{k+1}$. \\
                        Therefore, by definition of $S$, $t_1 \in S_{k+2}$.
                \end{itemize}
                Hence, $S_{i+1} \subseteq S_{i+2}$ holds.
        \end{itemize}
        From the above, since $S_i \subseteq S_{i+1}$ holds for each $i$, the set $S_i$ is cumulative.

    \vspace{10mm}

    \ansja{3.2.6}
        命題3.2.6の証明例が少々分かりにくかったので,私なりにまとめてみる. \\[5mm]
        $\Tau$ はある条件を満たす最小の集合と定義された.よって,\kakko{$a$} と \kakko{$b$} を示せば十分.
        \begin{itemize}
            \item [\kakko{$a$}] $S$ は条件を満たすこと
            \item [\kakko{$b$}] $S$ は条件を満たす集合の中で最小であること
        \end{itemize}
        ある条件:
        \begin{itemize}
            \item [\kakko{1}] $\{\texttt{true}, \; \texttt{false}, \; 0\} \in \Tau$
            \item [\kakko{2}] $t_1 \in \Tau$ ならば $\texttt{succ } t_1, \; \texttt{pred } t_1, \; \texttt{iszero } t_1 \in \Tau$
            \item [\kakko{3}] $t_1, t_2, t_3 \in \Tau$ ならば $\texttt{if } t_1 \texttt{ then } t_2 \texttt{ else } t_3 \in \Tau$
        \end{itemize}
        $\syomei$ 以下,証明を行う.
        \begin{itemize}
            \item [\kakko{$\boldsymbol{a}$}] {\bf $S$ は条件を満たすこと}
            \begin{itemize}
                \item $S_1 = \{\texttt{true}, \; \texttt{false}, \; 0\}$ より,条件\kakko{1}を満たす.
                \item ある $i$ が存在して,$t_1 \in S_i$ のとき,$S$ の定義より, \\
                    $\texttt{succ } t_1, \; \texttt{pred } t_1, \; \texttt{iszero } t_1 \in S_{i+1}$ であるから,条件\kakko{2}を満たす.
                \item ある $i$ が存在して,$t_1, t_2, t_3 \in S_i$ のとき,$S$ の定義より, \\
                    $\texttt{if } t_1 \texttt{ then } t_2 \texttt{ else } t_3 \in S_{i+1}$ であるから,条件\kakko{3}を満たす.
            \end{itemize}
            \item [\kakko{$\boldsymbol{b}$}] {\bf $S$ は条件を満たす集合の中で最小であること} \\[2mm]
            ある集合 $S^\prime$ が条件\kakko{1}, \kakko{2}, \kakko{3}を満たすとする.すべての $i$ に対して $S_i \subseteq S^\prime$ であることを示す.
            \begin{itemize}
                \item $i=0$ のとき \\[2mm]
                    $S_0 = \emptyset \subseteq S^\prime$ である.
                \item $i=1$ のとき \\[2mm]
                    $S_1 = \{\texttt{true}, \; \texttt{false}, \; 0\}$ であり,条件\kakko{1}より,$S_1 \subseteq S^\prime$ である.
                \item $i=k$ のとき,$S_i \subseteq S^\prime$ が成り立つと仮定して,$i=k+1$ のとき \\[2mm]
                    $t_1 \in S_{k+1}$ について
                    \begin{itemize}
                        \item $t_1$ が定数のとき \\[2mm]
                            $S$ の定義より,$t_1 \in S_k$ である.よって,帰納法の仮定より,$t_1 \in S^\prime$ である.
                        \item $t_1 = \texttt{succ } t_2$ または $t_1 = \texttt{pred } t_2$ または $t_1 = \texttt{iszero } t_2$ のとき \\[2mm]
                            $S$ の定義より, $t_2 \in S_k$ である.帰納法の仮定より,$S_k \subseteq S^\prime$ であるから,$t_2 \in S^\prime$ を得る.よって,条件\kakko{2}より,$t_1 \in S^\prime$ である.
                        \item $t_1 = \texttt{if } t_2 \texttt{ then } t_3 \texttt{ else } t_4$ のとき \\[2mm]
                            $S$ の定義より,$t_2, t_3, t_4 \in S_k$ である.帰納法の仮定より,$S_k \subseteq S^\prime$ であるから,$t_2, t_3, t_4 \in S^\prime$ を得る.よって,条件\kakko{3}より,$t_1 \in S^\prime$ である.
                    \end{itemize}
                したがって,すべての $i$ に対して $S_i \subseteq S^\prime$ であるから,$S = \displaystyle \bigcup_{i=0}^{\infty} S_i \subseteq S^\prime$ が成り立つ.
            \end{itemize}
            以上から,$S$ はある条件を満たす最小の集合である.
        \end{itemize}
        \kakko{$a$} と \kakko{$b$} が示されたので,$\Tau = S$ を得る.\qed

    \vspace{10mm}

    \ansen{3.2.6}
        Since the example proof of Proposition 3.2.6 was somewhat difficult to understand, I'll summarize it in my own way. \\[5mm]
        $\Tau$ was defined as the smallest set satisfying certain conditions. Therefore, it is sufficient to show \kakko{$a$} and \kakko{$b$}.
        \begin{itemize}
            \item [\kakko{$a$}] $S$ satisfies the conditions
            \item [\kakko{$b$}] $S$ is the smallest among sets satisfying the conditions
        \end{itemize}
        The conditions:
        \begin{itemize}
            \item [\kakko{1}] $\{\texttt{true}, \; \texttt{false}, \; 0\} \in \Tau$
            \item [\kakko{2}] if $t_1 \in \Tau$ then $\texttt{succ } t_1, \; \texttt{pred } t_1, \; \texttt{iszero } t_1 \in \Tau$
            \item [\kakko{3}] if $t_1, t_2, t_3 \in \Tau$ then $\texttt{if } t_1 \texttt{ then } t_2 \texttt{ else } t_3 \in \Tau$
        \end{itemize}
        $\syomeien$ In what follows, we give a proof.
        \begin{itemize}
            \item [\kakko{$\boldsymbol{a}$}] {\bf $S$ satisfies the conditions}
            \begin{itemize}
                \item Since $S_1 = \{\texttt{true}, \; \texttt{false}, \; 0\}$, it satisfies condition \kakko{1}.
                \item When there exists some $i$ such that $t_1 \in S_i$, by definition of $S$, \\
                    $\texttt{succ } t_1, \; \texttt{pred } t_1, \; \texttt{iszero } t_1 \in S_{i+1}$, thus satisfying condition \kakko{2}.
                \item When there exists some $i$ such that $t_1, t_2, t_3 \in S_i$, by definition of $S$, \\
                    $\texttt{if } t_1 \texttt{ then } t_2 \texttt{ else } t_3 \in S_{i+1}$, thus satisfying condition \kakko{3}.
            \end{itemize}
            \item [\kakko{$\boldsymbol{b}$}] {\bf $S$ is the smallest among sets satisfying the conditions} \\[2mm]
            Let $S^\prime$ be a set satisfying conditions \kakko{1}, \kakko{2}, \kakko{3}. We show that $S_i \subseteq S^\prime$ for all $i$.
            \begin{itemize}
                \item For $i=0$ \\[2mm]
                    $S_0 = \emptyset \subseteq S^\prime$.
                \item For $i=1$ \\[2mm]
                    Since $S_1 = \{\texttt{true}, \; \texttt{false}, \; 0\}$ and by condition \kakko{1}, $S_1 \subseteq S^\prime$.
                \item Assume $S_i \subseteq S^\prime$ holds for $i=k$, then for $i=k+1$ \\[2mm]
                    For $t_1 \in S_{k+1}$
                    \begin{itemize}
                        \item When $t_1$ is a constant \\[2mm]
                            By definition of $S$, $t_1 \in S_k$. Thus, by induction hypothesis, $t_1 \in S^\prime$.
                        \item When $t_1 = \texttt{succ } t_2$ or $t_1 = \texttt{pred } t_2$ or $t_1 = \texttt{iszero } t_2$ \\[2mm]
                            By definition of $S$, $t_2 \in S_k$. By induction hypothesis, since $S_k \subseteq S^\prime$, we get $t_2 \in S^\prime$. Therefore, by condition \kakko{2}, $t_1 \in S^\prime$.
                        \item When $t_1 = \texttt{if } t_2 \texttt{ then } t_3 \texttt{ else } t_4$ \\[2mm]
                            By definition of $S$, $t_2, t_3, t_4 \in S_k$. By induction hypothesis, since $S_k \subseteq S^\prime$, we get $t_2, t_3, t_4 \in S^\prime$. Therefore, by condition \kakko{3}, $t_1 \in S^\prime$.
                    \end{itemize}
                Thus, since $S_i \subseteq S^\prime$ holds for all $i$, we have $S = \displaystyle \bigcup_{i=0}^{\infty} S_i \subseteq S^\prime$.
            \end{itemize}
            From the above, $S$ is the smallest set satisfying the conditions.
        \end{itemize}
        Since \kakko{$a$} and \kakko{$b$} have been shown, we obtain $\Tau = S$.\qed

\end{document}