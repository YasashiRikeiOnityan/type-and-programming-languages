\documentclass[a4paper,11pt,fleqn,dvipdfmx]{article}

\usepackage[top=20mm,bottom=20mm,left=20mm,right=20mm]{geometry}

\usepackage{amsmath, tikz}
\newcommand{\ansja}[1]{\noindent\textbf{解答 #1}\\[2mm]}
\newcommand{\ansen}[1]{\noindent\textbf{Ans #1}\\[2mm]}
\newcommand{\maru}[1]{\hspace{-0.25em}\raisebox{-0.7ex}[0ex][0ex]{
    \tikz{
        \draw (0,0) circle (0.18cm);
        \draw (0,0) node {#1};
    }
}\hspace{-0.25em}}
\newcommand{\qed}[0]{\hfill\fbox{}}

\begin{document}

    \ansja{2.2.6}
        $R$ の反射的閉包を $R^{=}$ として,$R^{=} \subseteq R^\prime$ かつ $R^\prime \subseteq R^{=}$ を示す. \\
        $R^\prime = R \cup \{(s,s) \mid s \in S\}$ を\maru{1}とおく.
        \begin{itemize}
            \item $R^{=} \subseteq R^\prime$ の証明 \\[2mm]
            \maru{1}より,すべての $s \in S$ に対して $(s,s) \in R^\prime$ であるから $R^\prime$ は反射的である. \\
            また,$R^\prime$ は $R$ を含む.$R^{=}$ は $R$ を含む最小の反射的関係であるため,$R^{=} \subseteq R^\prime$ である.
            \item $R^\prime \subseteq R^{=}$の証明 \\[2mm]
            $(s,t) \in R^\prime$ とすると,\maru{1}より,$(s,t) \in R$ または $s=t \; (s \in S)$ である.
            \begin{itemize}
                \item $(s,t) \in R$ のとき,$R \subseteq R^{=}$ より,$(s,t) \in R^{=}$ である.
                \item $s=t \; (s \in S)$ のとき,$(s,t)=(s,s) \in R^{=}$ である.
            \end{itemize}
            したがって,$R^\prime \subseteq R^{=}$ である.
        \end{itemize}
        以上から,$R^\prime = R^{=}$ である.\qed

    \vspace{10mm}

    \ansen{2.2.6}
        Let $R^{=}$ be the reflexive closure of $R$. We show that $R^{=} \subseteq R^\prime$ and $R^\prime \subseteq R^{=}$. \\
        Let $R^\prime = R \cup \{(s,s) \mid s \in S\}$ be \maru{1}.
        \begin{itemize}
            \item Proof of $R^{=} \subseteq R^\prime$ \\[2mm]
            By \maru{1}, $R^\prime$ is reflexive since $(s,s) \in R^\prime$ for all $s \in S$. \\
            Moreover, $R^\prime$ contains $R$. Since $R^{=}$ is the minimal reflexive relation containing $R$, we have $R^{=} \subseteq R^\prime$.
            \item Proof of $R^\prime \subseteq R^{=}$ \\[2mm]
            Suppose $(s,t) \in R^\prime$. Then by \maru{1}, either $(s,t) \in R$ or $s=t \; (s \in S)$.
            \begin{itemize}
            \item If $(s,t) \in R$, then $(s,t) \in R^{=}$ since $R \subseteq R^{=}$.
            \item If $s=t \; (s \in S)$, then $(s,t)=(s,s) \in R^{=}$.
            \end{itemize}
            Therefore, $R^\prime \subseteq R^{=}$.
        \end{itemize}
        Hence, $R^\prime = R^{=}$.\qed

    \pagebreak

    \ansja{2.2.7}
        $R^{T}$ を $R$ の推移的閉包とする. $R^{T} \subseteq R^{+}$ と $R^{+} \subseteq R^{T}$ を示す.
        \begin{itemize}
            \item $R^{T} \subseteq R^{+}$ \\[2mm]
            $(s,t),(t,u) \in R^{+}$ とすると,ある $i,j$ が存在して $(s,t) \in R_i$ かつ $(t,u) \in R_j$ である. \\
            $R^{+}$ の定義より $(s,u) \in R_{\mathrm{max}(i,j)+1}$ であり,したがって,$(s,u) \in R^{+}$. \\
            よって,$R^{+}$ は推移的である. \\
            また,$R^{+}$ は $R$ を含む.$R^{T}$ は $R$ を含む最小の推移的関係であるため,$R^{T} \subseteq R^{+}$ である.
            \item $R^{+} \subseteq R^{T}$ \\[2mm]
            任意の $i$ について $R_i \subseteq R^{T}$ であることを数学的帰納法で示す.
            \begin{itemize}
                \item $i=0$ のとき,$R_0 = R \subseteq R^{T}$.
                \item $i=n$ のとき$R_i \subseteq R^{T}$ が成り立つと仮定して,$i=n+1$ のとき \\
                $(s,u) \in R_{n+1}$ とすると,ある $t \in S$ が存在して $(s,t),(t,u) \in R_n$ である. \\
                帰納法の仮定より $(s,t),(t,u) \in R^{T}$ であり,$R^{T}$ は推移的なので $(s,u) \in R^{T}$.
            \end{itemize}
            したがって,$R^{+} \subseteq R^{T}$ である.
        \end{itemize}
        以上から,$R^{+} = R^{T}$ である.\qed

    \vspace{10mm}

    \ansen{2.2.7}
        Let $R^{T}$ be the transitive closure of $R$. We show that $R^{T} \subseteq R^{+}$ and $R^{+} \subseteq R^{T}$.
        \begin{itemize}
            \item $R^{T} \subseteq R^{+}$ \\[2mm]
                Suppose $(s,t), (t,u) \in R^{+}$. Then there exist some $i,j$ such that $(s,t) \in R_i$ and $(t,u) \in R_j$. \\
                By the definition of $R^{+}$, we have $(s,u) \in R_{\mathrm{max}(i,j)+1}$, and therefore $(s,u) \in R^{+}$. \\
                Thus, $R^{+}$ is transitive. \\
                Moreover, $R^{+}$ contains $R$. Since $R^{T}$ is the minimal transitive relation containing $R$, we have $R^{T} \subseteq R^{+}$.
            \item $R^{+} \subseteq R^{T}$ \\[2mm]
                We prove that $R_i \subseteq R^{T}$ for all $i$ by mathematical induction.
                \begin{itemize}
                    \item When $i=0$, $R_0 = R \subseteq R^{T}$.
                    \item Assuming $R_i \subseteq R^{T}$ holds for $i=n$, let's prove for $i=n+1$. \\
                        If $(s,u) \in R_{n+1}$, then there exists some $t \in S$ such that $(s,t),(t,u) \in R_n$. \\
                        By the induction hypothesis, $(s,t),(t,u) \in R^{T}$, and since $R^{T}$ is transitive, $(s,u) \in R^{T}$.
                \end{itemize}
                Therefore, $R^{+} \subseteq R^{T}$.
        \end{itemize}
        Hence, $R^{+} = R^{T}$.\qed

\end{document}