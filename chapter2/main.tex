\documentclass[a4paper,11pt,fleqn,dvipdfmx]{article}

\usepackage[top=20mm,bottom=20mm,left=20mm,right=20mm]{geometry}

\usepackage{amsmath, tikz}
\newcommand{\ansja}[1]{\noindent\textbf{回答 #1}\\[2mm]}
\newcommand{\ansen}[1]{\noindent\textbf{Ans #1}\\[2mm]}
\newcommand{\maru}[1]{\hspace{-0.25em}\raisebox{-0.7ex}[0ex][0ex]{
    \tikz{
        \draw (0,0) circle (0.18cm);
        \draw (0,0) node {#1};
    }
}\hspace{-0.25em}}
\newcommand{\owari}[0]{\hfill\fbox{}}

\begin{document}

    \ansja{2.2.6}
        $R$ の反射的閉包を $R^{=}$ として,$R^{=} \subseteq R^\prime$ かつ $R^\prime \subseteq R^{=}$ を示す. \\
        $R^\prime = R \cup \{(s,s) \mid s \in S\}$ を\maru{1}とおく.
        \begin{enumerate}
            \item $R^{=} \subseteq R^\prime$ \\[2mm]
            \maru{1}より,すべての $s \in S$ に対して $(s,s) \in R^\prime$ であるから $R^\prime$ は反射的である. \\
            また,$R^\prime$ は $R$ を含むため,$R^{=} \subseteq R^\prime$ である.
            \item $R^\prime \subseteq R^{=}$ \\[2mm]
            $(s,t) \in R^\prime$ とすると,\maru{1}より,$(s,t) \in R$ または $s=t$ である.
            \begin{itemize}
                \item $(s,t) \in R$ のとき,$R \subseteq R^{=}$ より,$(s,t) \in R^{=}$ である.
                \item $s=t$ のとき,$(s,t)=(s,s) \in R^{=}$ である.
            \end{itemize}
            したがって,$R^\prime \subseteq R^{=}$ である.
        \end{enumerate}
        以上から, $R^\prime = R^{=}$ である.\owari

    \vspace{10mm}

    \ansen{2.2.6}
        Let $R^{=}$ be the reflexive closure of $R$. We show that $R^{=} \subseteq R^\prime$ and $R^\prime \subseteq R^{=}$. \\
        Let $R^\prime = R \cup \{(s,s) \mid s \in S\}$ be \maru{1}.
        \begin{enumerate}
            \item $R^{=} \subseteq R^\prime$ \\[2mm]
            By \maru{1}, we have $(s,s) \in R^\prime$ for all $s \in S$, so $R^\prime$ is reflexive. \\
            Since $R^\prime$ contains $R$, we have $R^{=} \subseteq R^\prime$.
            \item $R^\prime \subseteq R^{=}$ \\[2mm]
            Suppose $(s,t) \in R^\prime$. Then by \maru{1}, either $(s,t) \in R$ or $s=t$.
            \begin{itemize}
                \item If $(s,t) \in R$, then $(s,t) \in R^{=}$ since $R \subseteq R^{=}$.
                \item If $s=t$, then $(s,t)=(s,s) \in R^{=}$.
            \end{itemize}
            Therefore, $R^\prime \subseteq R^{=}$.
        \end{enumerate}
        Hence, $R^\prime = R^{=}$.\owari

    \pagebreak

    \ansja{2.2.7}
        $R^{T}$を$R$の推移的閉包とする. $R^{T} \subseteq R^{+}$と$R^{+} \subseteq R^{T}$を示す.
        \begin{enumerate}
            \item $R^{T} \subseteq R^{+}$ \\[2mm]
            $(s,t),(t,u) \in R^{+}$とすると,ある$i,j$が存在して$(s,t) \in R_i$かつ$(t,u) \in R_j$である. \\
            $R^{+}$の定義より$(s,u) \in R_{\mathrm{max}(i,j)+1}$であり,したがって$(s,u) \in R^{+}$. \\
            よって$R^{+}$は推移的である. \\
            また$R^{+}$は$R$を含むので,$R^{T} \subseteq R^{+}$である.
            \item $R^{+} \subseteq R^{T}$ \\[2mm]
            任意の$i$について$R_i \subseteq R^{T}$を数学的帰納法で示す.
            \begin{itemize}
                \item $i=0$のとき,$R_0 = R \subseteq R^{T}$.
                \item $i=n$のとき$R_n \subseteq R^{T}$と仮定する. \\
                $i=n+1$のとき,$(s,u) \in R_{n+1}$とすると,ある$t \in S$が存在して$(s,t),(t,u) \in R_n$.
                帰納法の仮定より$(s,t),(t,u) \in R^{T}$であり,$R^{T}$が推移的なので$(s,u) \in R^{T}$.
            \end{itemize}
            よって$R^{+} \subseteq R^{T}$.
        \end{enumerate}
        以上より$R^{+} = R^{T}$.\owari

    \vspace{10mm}

    \ansen{2.2.7}
        Let $R^{T}$ be the transitive closure of $R$. We show that $R^{T} \subseteq R^{+}$ and $R^{+} \subseteq R^{T}$.
        \begin{enumerate}
            \item $R^{T} \subseteq R^{+}$ \\[2mm]
            Suppose $(s,t), (t,u) \in R^{+}$. Then there exist some $i,j$ such that $(s,t) \in R_i$ and $(t,u) \in R_j$. \\
            By the definition of $R^{+}$, we have $(s,u) \in R_{\mathrm{max}(i,j)+1}$, so $(s,u) \in R^{+}$. \\
            Therefore, $R^{+}$ is transitive. \\
            Since $R^{+}$ contains $R$, we have $R^{T} \subseteq R^{+}$.
            \item $R^{+} \subseteq R^{T}$ \\[2mm]
            We prove that $R_i \subseteq R^{T}$ for all $i$ by mathematical induction.
            \begin{itemize}
                \item When $i=0$, we have $R_0 = R \subseteq R^{T}$.
                \item Assume $R_n \subseteq R^{T}$ for $i=n$. \\
                When $i=n+1$, suppose $(s,u) \in R_{n+1}$. Then there exists some $t \in S$ such that $(s,t),(t,u) \in R_n$.
                By the induction hypothesis, $(s,t),(t,u) \in R^{T}$, and
                since $R^{T}$ is transitive, we have $(s,u) \in R^{T}$.
            \end{itemize}
            Therefore, $R^{+} \subseteq R^{T}$.
        \end{enumerate}
        Hence, $R^{+} = R^{T}$.\owari
        
    
\end{document}